\section{Part 3: Tax allowance for seniors}
\label{sec:4_part_three}

This part of the assignment introduces a tax relief program for people aged 65 years and older. It increases their allowed tax-free income to 7000. The main challenge of this task is the integration of this additional rule into the existing tax allowance logic.

%-------------------------------------------------------------------------
\subsection{Invariants}

To model the constraints given by the task, two additional invariants were defined. The first invariant in Listing \ref{lst:senior-tax-allowance} checks that the tax allowance of an unmarried \texttt{Taxpayer} is exactly 7000 at all times.

\noindent
\begin{minipage}{\linewidth}
  \lstinputlisting[ language=Solidity, firstline=117, lastline=122, caption={Tax allowance of a senior taxpayer should always be 7000 invariant in echidna.sol}, label={lst:senior-tax-allowance} ]{../../echidna.sol}
\end{minipage}

The second invariant checks that the combined allowance of a married couple over 64 is exactly 14000, as both spouses should get the extended tax allowance.

\noindent
\begin{minipage}{\linewidth}
  \lstinputlisting[ language=Solidity, firstline=127, lastline=136, caption={Tax allowance of a senior married couple should always be 14000 invariant in echidna.sol}, label={lst:senior-tax-allowance-combined} ]{../../echidna.sol}
\end{minipage}

Of course, the tax allowance of elderly taxpayers should also never get below zero, so the invariant in Listing \ref{lst:tax-allowance-above-zero} defined in task 2 also applies to this task.

By default, every \texttt{Taxpayer} is initialized with an \texttt{age} variable and the function \texttt{haveBirthday()}. The function progresses the age of the \texttt{Taxpayer} by one year. Every \texttt{Taxpayer} is "born" by default with the age zero, so the function needs to be called at least 65 times in order for the extended tax allowance to take effect. To increase the likelihood of Echidna reaching that threshold during testing, two new supporting functions were also added.

\noindent
\begin{minipage}{\linewidth}
  \lstinputlisting[ language=Solidity, firstline=141, lastline=145, caption={Advance Taxpayer age by 15 years in echidna.sol}, label={lst:advance-age-quickly} ]{../../echidna.sol}
\end{minipage}

\noindent
\begin{minipage}{\linewidth}
  \lstinputlisting[ language=Solidity, firstline=150, lastline=154, caption={Advance spouses age by 15 years in echidna.sol}, label={lst:advance-age-quickly-spouse} ]{../../echidna.sol}
\end{minipage}

The supporting functions shown in Listing \ref{lst:advance-age-quickly} and \ref{lst:advance-age-quickly-spouse} increase the age of both the main \texttt{Taxpayer} and its spouse in 15-year increments. That way, Echidna only needs to call this function five times total to reach the extended tax allowance of the two individuals.

%-------------------------------------------------------------------------
\subsection{Code improvements}

Since the default code has no inherent logic for setting the extended tax allowance for elderly \texttt{Taxpayers}, the \texttt{haveBirthday()} function was adapted with a condition for the age of the respective taxpayer.

\noindent
\begin{minipage}{\linewidth}
  \lstinputlisting[ language=Solidity, firstline=120, lastline=126, caption={Increase age on birthday in Taxpayer.sol}, label={lst:advance-birthday} ]{../../Taxpayer.sol}
\end{minipage}

The function referenced in Listing \ref{lst:advance-birthday} checks, in addition to increasing the age of the \texttt{Taxpayer}, if it crossed the age of 65 and has not received its extended tax allowance yet. If the checks pass, a setter function is called to set a boolean value \texttt{extendedTaxAllowance} permanently to \texttt{true}.

\noindent
\begin{minipage}{\linewidth}
  \lstinputlisting[ language=Solidity, firstline=146, lastline=154, caption={Set and get extended tax allowance in Taxpayer.sol}, label={lst:set-extended-tax-allowance} ]{../../Taxpayer.sol}
\end{minipage}

The setter as shown in Listing \ref{lst:set-extended-tax-allowance} is purposefully made to only set the \texttt{extendedTaxAllowance} to \texttt{true} once. It reflects the fact that it is not possible for an individual to lose its extended tax allowance or increase it multiple times. The extended tax allowance can be gained indefinitely either by winning the lottery or reaching the age of 65. In addition to setting the \texttt{extendedTaxAllowance} boolean, it also adds the difference between the default and extended tax allowance to the individual's tax allowance once. This approach is preferable to setting the tax allowance back to 7000, because it prevents cases where a spouse already transferred some of its initial tax allowance and therefore the couple would gain more than 2000 new tax allowance in total.

\noindent
\begin{minipage}{\linewidth}
  \lstinputlisting[ language=Solidity, firstline=136, lastline=138, caption={Add tax allowance in Taxpayer.sol}, label={lst:add-tax-allowance} ]{../../Taxpayer.sol}
\end{minipage}

For this reason, the \texttt{private} helper function \texttt{addTaxAllowance()} (Listing \ref{lst:add-tax-allowance}) was added. If two \texttt{Taxpayers} divorce, the extended tax allowance must return to the \texttt{Taxpayer}(s) that won the lottery in the first place. To ensure that this is the case, the \texttt{divorce()} function was modified to reset each spouses tax allowance correctly (see Listing \ref{lst:divorce}):

\noindent
\begin{minipage}{\linewidth}
  \lstinputlisting[ language=Solidity, firstline=78, lastline=93, caption={Divorce married couples in Taxpayer.sol}, label={lst:divorce} ]{../../Taxpayer.sol}
\end{minipage}

Regarding additional security concerns, a similar access control problem to part 2 can be identified. Since the \texttt{haveBirthday()} function is public with no further checks for owner privileges, anyone could increase the birthday of an individual taxpayer provided he or she has access to their address. Especially in the case of tracking the age of an individual, the progression of birthdays should ideally be handled in a fully automatic process. Additionally, it would be beneficial to track the exact birthdate of a \texttt{Taxpayer} in order to allow for an accurate age tracking.